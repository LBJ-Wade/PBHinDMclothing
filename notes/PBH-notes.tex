\documentclass[a4paper,11pt]{article}
%\pdfoutput=1

\usepackage{amssymb,amsmath,bm}
\usepackage{a4wide}
%\usepackage{cite}
\usepackage{color}
\usepackage{slashed}
\usepackage{graphicx}
\usepackage{amsfonts}
\usepackage{lscape}
\usepackage{amsthm}
\usepackage{booktabs}
\usepackage{array}
\usepackage{rotating}
\usepackage[numbers, sort]{natbib}
\usepackage{cancel}
%\usepackage{multirow}
\usepackage{float}
\usepackage{xspace}
\usepackage[colorlinks=true,urlcolor=blue,anchorcolor=blue,citecolor=blue,filecolor=blue,linkcolor=blue,menucolor=blue,pagecolor=blue]{hyperref}
\usepackage[utf8]{inputenc}

\newcommand{\Bradley}[1]{{\color{red} \textbf{BJK:} #1}}

\bibliographystyle{JHEP}

\usepackage[small,bf]{caption}
\setlength{\captionmargin}{\parindent}

%%%%%%%%%%%%%%%%%%%%%%%%%%%%%%%%%%%%%%%%%%%%%%%%%%%%%%%%%%%%%%%%%%%%%%
%Global Defs

\newcommand{\GeV}{{\, \rm GeV}}
\newcommand{\TeV}{{\, \rm TeV}}
\newcommand{\eps}{\epsilon}
\newcommand{\ord}[1]{\mathcal{O}\left( #1 \right)}
\newcommand{\gm}{\gamma^\mu}
\newcommand{\be}{\begin{equation}}
\newcommand{\ee}{\end{equation}}
\newcommand{\vev}[1] {\langle #1 \rangle}

\newcommand{\ONR}[1]{\mathcal{O}_{#1}^{\mathrm{NR}}}
\newcommand{\OLR}[1]{\mathcal{O}_{#1}^{\mathrm{LR}}}
\newcommand{\CEvNS}{CE$\nu$NS\xspace}
\newcommand{\rperi}{r_\mathrm{peri}}

%%%%%%%%%%%%%%%%%%%%%%%%%%%%%%%%%%%%%%%%%%%%%%%%%%%%%%%%%%%%%%

\begin{document}

\title{Primordial Black Holes (in Dark Matter clothing)}
\date{\today}
\maketitle

See the jupyter notebook or plots folder for some visualisations.

\section{Length Scales}

\subsection{Distribution of orbits}

From Ref.~\cite{Sasaki:2016jop} (Sec.~2), the probability distribution for the semi-major axis $a$ and the eccentricity $e$ of the binary PBH orbits is given by:
\begin{equation}
\mathrm{d}P(a,e) = \frac{3}{4} f^{3/2} \bar{x}^{-3/2} a^{1/2} e (1-e^2)^{-3/2}\,\mathrm{d}a\,\mathrm{d}e\,.
\end{equation}
Here, $f$ is the fraction of PBHs in DM and $\bar{x}$ is the mean physical separation of PBHs at matter-radiation equality $z = z_\mathrm{eq}$:
\begin{equation}
\overline{x} = \frac{1}{(1+z_\mathrm{eq}) f^{1/3}} \left(\frac{8\pi G}{3 H_0^2} \frac{M_\mathrm{BH}}{\Omega_{DM}}\right)^{1/3}\,.
\end{equation}
Note that there could be some extra factors of $4\pi/3$ in here, depending on your precise definition of $\bar{x}$ (see e.g.~Eq.~2 in Ref.~\cite{Ali-Haimoud:2017rtz}).


\subsection{Range of parameter values}

The maximum eccentricity is given by (Ref.~\cite{Sasaki:2016jop}, Eq.~6):
\begin{equation}
e_\mathrm{max} = \sqrt{1 - f^{3/2} \left(\frac{a}{\bar{x}}\right)^{3/2}}\,.
\end{equation}

The distance of closest approach of the PBHs - \textit{peri-BH}, $r_\mathrm{peri}$ - is given by:
\begin{equation}
r_\mathrm{peri} = a (1-e)\,.
\end{equation}
Then periBH lies in the range:
\begin{equation}
\rperi \in a[1 - e_\mathrm{max}, 1]\,.
\end{equation}
We note also that the requirement for forming a binary is that the physical separation of the BH is
\begin{equation}
x < f^{1/3} \bar{x}\,,
\end{equation}
which means that the semi-major axis is limited to:
\begin{equation}
a \leq \alpha f^{1/3} \bar{x}\,,
\end{equation}
where $\alpha$ is a numerical factor of $\mathcal{O}(1)$, set to 1 in Ref.~\cite{Sasaki:2016jop}.

\subsection{Distribution of  (dimensionless) periBH}

We now change variables from eccentricity to periBH:
\begin{equation}
\label{eq:PDF_periBH}
\mathrm{d}P(a, \rperi) = \frac{3}{4} f^{3/2} \bar{x}^{-3/2} a^{-1/2} \left(1-\frac{\rperi}{a}\right)\left(\frac{2\rperi}{a} - \frac{\rperi^2}{a^2}\right)^{-3/2}\,\mathrm{d}a\,\mathrm{d}\rperi\,.
\end{equation}
Or, in terms of the dimensionless periBH $\mathfrak{r} = \rperi/a$:
\begin{equation}
\mathrm{d}P(a, \mathfrak{r}) = \frac{3}{4} f^{3/2} \bar{x}^{-3/2} a^{1/2} \left(1-\mathfrak{r}\right)\left(2 \mathfrak{r} - \mathfrak{r}^2\right)^{-3/2}\,\mathrm{d}a\,\mathrm{d}\mathfrak{r}\,.
\end{equation}
Note here, that the PDF is valid over the range:
\begin{align}
a &\in [0, f^{1/3} \bar{x}]\,,\\
\mathfrak{r} &\in [1 - e_\mathrm{max}(a), 1]\,.
\end{align}
We can rearrange the limits:
\begin{align}
a &\in [0, \frac{\bar{x}}{f} \left(1 - (1-\mathfrak{r})^2\right)^{2/3}]\,,\\
\mathfrak{r} &\in [0, 1]\,,
\end{align}
and then integrate over $a$:
\begin{equation}
\mathrm{d}P(\mathfrak{r}) = \frac{1}{2} \frac{\left(1-\mathfrak{r}\right)}{\sqrt{1- (1-\mathfrak{r})^2}}\,\mathrm{d}\mathfrak{r}\,.
\end{equation}
As a sanity check, we can integrate over all $\mathfrak{r} \in [0, 1]$ and we obtain $1/2$. This matches the original normalisation of the PDFs from Ref.~\cite{Sasaki:2016jop} (my guess is that they normalised to 1/2 because one PBH is one-half of a PBH binary).


\subsection{Sensible units}

The density of PBHs today is \cite{Lahav:2014vza,Ade:2015xua}
\begin{equation}
\rho_{\mathrm{PBH}} = \rho_\mathrm{crit} f \Omega_{\mathrm{DM}} = 3.3 \times 10^{10} \, f M_\odot \, \mathrm{Mpc}^{-3}\,,
\end{equation}
Taking matter-radiation equality to be at $z_\mathrm{eq} \approx 3400$ \cite{Ade:2015xua}, we obtain
\begin{equation}
\bar{x} \approx 3 \times 10^{-1} \left( \frac{M_{\mathrm{PBH}}}{30 \,M_\odot}\right)^{1/3}{f}^{-1/3} \, \mathrm{pc}\,.
\end{equation}

The Schwarzschild radius of the PBH is:
\begin{equation}
r_s \approx 3 \times 10^{-12} \left( \frac{M_{\mathrm{PBH}}}{30 \,M_\odot}\right)\,\mathrm{pc}\,.
\end{equation}

\subsection{Dark Matter Clothing}

As an initial estimate of the size of a DM halo around a PBH, we can take the truncation radius at $z \approx z_\mathrm{eq}$ given by Eq.~1 of Ref.~\cite{Lacki:2010zf} (see also Refs.~\cite{Mack:2006gz,Ricotti:2007jk}):
\begin{equation}
R_\mathrm{tr}(z) = 2 \times 10^{-2} \, \mathrm{pc} \left( \frac{M_{\mathrm{PBH}}}{30 \, M_\odot} \right)^{1/3} \left( \frac{1+z_{eq}}{1+z}\right)\,.
\end{equation}

\subsection{Classifying different regimes}

We can classify 3 regimes:
\begin{itemize}
\item \textbf{Isolated binaries}, where $r_\mathrm{peri}$ is always much larger than the halo truncation radius\,,
\item \textbf{Common-halo binaries}, where $r_\mathrm{peri}$ and $a$ are both smaller than the halo truncation radius\,,
\item \textbf{Close-passage binaries}, where $r_\mathrm{peri}$ is below the halo truncation radius, but $a$ is above the halo truncation radius\,.
\end{itemize}

\textit{Check the jupyter notebook for some calculations and plots.} The overall result seems to be that \textbf{close-passage binaries} are the dominant population (for $f \lesssim 0.1$).

\bibliographystyle{JHEP}
\bibliography{PBH.bib}

\end{document}